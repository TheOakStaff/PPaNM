\documentclass{article}
\usepackage{graphicx}
\title{A brief introduction to \LaTeX document preparation system}
\author{C.S.O.}
\date{}
\begin{document}
\maketitle
\begin{abstract}
	We show some basic capabilities of the \LaTeX system: math, figures, references, sections...
\end{abstract}
\section{Math}
The gamma-function, denoted $\Gamma$, is usually defined via an integral,
	\begin{equation}\label{eq:integ}
\Gamma(z)=\int_0^\infty x^{z-1}e^{-x} dx\;
	\end{equation}

\section{Numerical approximation of the gamma function}
One of many simple approximations to the integral~(\ref{eq:integ}) function is the Gergo Nemes~\cite{gergo-nemes} formulae,
	\begin{equation}\lable{eq:nemes}
\Gamma(z) \approx \sqrt{\frac{2\pi}{z} } \left(\frac{1}{e} \left(z +
\frac{1}{12z - \frac{1}{10z}}\right)\right)^z \;
	\end{equation}

\section{Figures}
Here is an illustration of the Nemes' formula, compared to the built-in gamma function from the standart C-library. First, the "\LaTeX" terminal of Gnuplot.
	\begin{figure}\lable{fig:gpl}
\input{fig-gpl.tex}
	\end{figure}


\begin{thebibliography}{9}
\bibitem{gergo-nemes} Nemes, Gergo (2010), "New asymptotic expansion
for the Gamma function", Archiv der Mathematik, 95 (2): 161–169,
\end{thebibliography}{9}
\end{document}
